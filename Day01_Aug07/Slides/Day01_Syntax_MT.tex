\documentclass[xcolor=x11names,compress]{beamer}

%% General document %%%%%%%%%%%%%%%%%%%%%%%%%%%%%%%%%%
\PassOptionsToPackage{table}{xcolor}
\usepackage{graphicx}
\usepackage{multirow,multicol}
\usepackage{amsmath}
\usepackage{mathpazo}
\usepackage{amsthm}
\usepackage{amssymb}
\usepackage{setspace}
\usepackage{hyperref}
\usepackage{array,colortbl,booktabs}
\usepackage{soul}
\usepackage{enumerate}
\usepackage{url}
\usepackage{verbatimbox}
\usepackage{fancyvrb}
\usepackage{dirtree}
\usepackage{tikz}
\usetikzlibrary{positioning,shapes.misc}
\usetikzlibrary{decorations.fractals}
\usetikzlibrary{calc}
\usepackage[normalem]{ulem}
\useunder{\uline}{\ul}{}
\usetikzlibrary{arrows}
\usetikzlibrary{fit}
%\usepackage{color, colortbl}
\usepackage{etoolbox}
\makeatletter
\patchcmd{\slideentry}{\advance\beamer@xpos by1\relax}{}{}{}
\def\beamer@subsectionentry#1#2#3#4#5{\advance\beamer@xpos by1\relax}%
\makeatother
%%%%%%%%%%%%%%%%%%%%%%%%%%%%%%%%%%%%%%%%%%%%%%%%%%%%%%


%% Beamer Layout %%%%%%%%%%%%%%%%%%%%%%%%%%%%%%%%%%
\useoutertheme[subsection=false, shadow]{miniframes}
\useinnertheme{circles}
\usefonttheme{structurebold}
\usepackage{palatino}
\usepackage{tcolorbox}
\usepackage{lipsum}

%% COLOR DEFINITION %%%%%%%%%%%%%%%%%%%%%%%%%%%%%%%%
\definecolor{mygold}{RGB}{236,208,120}
\definecolor{brick}{RGB}{217,91,67}
\definecolor{myred}{RGB}{192,41,66}
\definecolor{redwine}{RGB}{102,0,102}
\definecolor{myaqua}{RGB}{83,119,122}

%%%%%%%%%%%%%%%%%%%%%%%%%%%%%%%%%%%%%%%%%%%%%%%%%%%%%%

%%% OTHER EXTRA STUFF%%%%%%%%%%%%%%%%%%%%%%%%%%%%%%%%
%% CHANGE COVER SLIDE %%
\makeatletter
    \newenvironment{coverframe}{
\setbeamercolor*{palette tertiary}{fg=myaqua,bg=myaqua} 
}
    {}
\makeatother

%% CHANGE COLOR OF BULLETS %%
\setbeamercolor{item}{fg=brick} % color of bullets
\setbeamercolor{subitem}{fg=mygold}

%% DEFINE TIKZ ITEMS %%
\tikzset{
    %Define standard arrow tip
    >=stealth',
    %Define style for boxes
    punkt/.style={
           rectangle,
           rounded corners,
           draw=black, very thick,
           text width=3em,
           minimum height=2em,
           text centered},
    % Define arrow style 1
    pil/.style={
           <-,
           thick,
           shorten <=2pt,
           shorten >=2pt,}
        % Define arrow style
           }
           
 %% Perpendiculas symbol
 \newcommand\independent{\protect\mathpalette{\protect\independenT}{\perp}}
\def\independenT#1#2{\mathrel{\rlap{$#1#2$}\mkern2mu{#1#2}}}

\newcommand{\ds}{\displaystyle}

\newcommand{\bv}{\begin{Verbatim}[numbers=left, baselinestretch=1,
    xleftmargin=.5in, xrightmargin=.1in, frame=single,
    rulecolor=\color{gray}]}

%%%%%%%%%%%%%%%%%%%%%%%%%%%%%%%%%%%%%%%%%%%%%%%%%%
\setbeamerfont{title like}{shape=\scshape}
\setbeamerfont{frametitle}{shape=\scshape}
\setbeamercolor{frametitle}{fg=myaqua!120,bg=white}
\setbeamercolor{title}{fg=myaqua!120}
%\setbeamertemplate{frametitle}{\vspace{6em}}

\setbeamercolor*{lower separation line head}{bg=redwine} 
\setbeamercolor*{normal text}{fg=black} 
\setbeamercolor*{alerted text}{fg=myred} 
\setbeamercolor*{example text}{fg=brick} 
\setbeamercolor*{structure}{fg=black} 
 
\setbeamercolor*{palette tertiary}{fg=black!80,bg=white} 
\setbeamercolor*{palette quaternary}{fg=black,bg=black!10} 

\renewcommand{\(}{\begin{columns}}
\renewcommand{\)}{\end{columns}}
\newcommand{\<}[1]{\begin{column}{#1}}
\renewcommand{\>}{\end{column}}

%%%%%%%%%%%%%%%%%%%%%%%%%%%%%%%%%%%%%%%%%%%%%%%%%%
\title{Introduction to Python}
\author{Michelle Torres}
\date{August 7, 2016}

\begin{document}
\newcommand<>{\highlighton}[1]{%
  \alt#2{\structure{#1}}{{#1}}
}

\newcommand{\icon}[1]{\pgfimage[height=1em]{#1}}

%%%%%%%%%%%%%%%%%%%%%%%%%%%%%%%%%%%%%%%%%%%%%%%%%%
%%%%%%%%%%%%%%%%%%%%%%%%%%%%%%%%%%%%%%%%%%%%%%%%%%
%%%%%%%% START THE SLIDES %%%%%%%%%%%%%%%%%%%%%%%%
%%%%%%%%%%%%%%%%%%%%%%%%%%%%%%%%%%%%%%%%%%%%%%%%%%
%%%%%%%%%%%%%%%%%%%%%%%%%%%%%%%%%%%%%%%%%%%%%%%%%%

\begin{coverframe}
\begin{frame}
\titlepage	
\end{frame}
\end{coverframe}


\section{Introduction}
%%%%%%%%%%%%%%%%%%%%%%%%%%%%%%%%%%%%%%%%%%%%%%%%%%
%% Course Overview
%%%%%%%%%%%%%%%%%%%%%%%%%%%%%%%%%%%%%%%%%%%%%%%%%%
\subsection{Course overview}
\begin{frame}
\frametitle{Course Overview}
\begin{itemize}
\item Michelle's office hours (277):
\begin{itemize}
\item Officially one hour after every class meeting
\item Feel free to stop by any time I'm in
\item Email questions or if you want to meet
\end{itemize}
\item Homeworks:
\begin{itemize}
\item Will be about 6 homework assignments
\item Will be due Thursday and Monday (end of day)
\item Can work together, but each keystroke should be your own
\item All work must be done on git -- commit often with comments
\item Direct all questions about grading, due date, etc. to Erin
\end{itemize}
\item Poster session TBD
\end{itemize}
\end{frame}
%%%%%%%%%%%%%%%%%%%%%%%%%%%%%%%%%%%%%%%%%%%%%%%%%%


%%%%%%%%%%%%%%%%%%%%%%%%%%%%%%%%%%%%%%%%%%%%%%%%%%
%% Goals
%%%%%%%%%%%%%%%%%%%%%%%%%%%%%%%%%%%%%%%%%%%%%%%%%%
\subsection{Goals}
\begin{frame}
\frametitle{Goals}
\begin{itemize}
\item Learn Python
\begin{itemize}
\item Web scraping, APIs, data structures, etc.
\end{itemize}
\item Transferable skills to other languages
\begin{itemize}
\item Ruby, SQL, Perl, programming logic
\end{itemize}
\item Send a signal!
\end{itemize}
\end{frame}
%%%%%%%%%%%%%%%%%%%%%%%%%%%%%%%%%%%%%%%%%%%%%%%%%%


%%%%%%%%%%%%%%%%%%%%%%%%%%%%%%%%%%%%%%%%%%%%%%%%%%
%% Quiz
%%%%%%%%%%%%%%%%%%%%%%%%%%%%%%%%%%%%%%%%%%%%%%%%%%
\subsection{Quiz}
\begin{frame}
 \frametitle{Quiz (!)}
 \begin{itemize}
 	\item Please go to:
 	\begin{itemize}
 	\item http://smtorres.org/quiz1.html
	\item http://smtorres.org/quiz2.html	
	\end{itemize}
 \end{itemize}
\end{frame}
%%%%%%%%%%%%%%%%%%%%%%%%%%%%%%%%%%%%%%%%%%%%%%%%%%

\section{Syntax}
%%%%%%%%%%%%%%%%%%%%%%%%%%%%%%%%%%%%%%%%%%%%%%%%%%
%% Syntax
%%%%%%%%%%%%%%%%%%%%%%%%%%%%%%%%%%%%%%%%%%%%%%%%%%
\begin{frame}
\frametitle{Syntax}
\begin{itemize}
\item Object types
\begin{itemize}
\item String
\item Int
\item Float
\item List
\item Tuple
\item Dictionary
\end{itemize}
\item Conditionals
\item Loop
\item Functions
\end{itemize}
\end{frame}
%%%%%%%%%%%%%%%%%%%%%%%%%%%%%%%%%%%%%%%%%%%%%%%%%%

%%%%%%%%%%%%%%%%%%%%%%%%%%%%%%%%%%%%%%%%%%%%%%%%%%
%% Strings 1
%%%%%%%%%%%%%%%%%%%%%%%%%%%%%%%%%%%%%%%%%%%%%%%%%%
\subsection{Strings}
\begin{frame}[fragile]
\frametitle{Strings}
\begin{itemize}
\item Any group of characters recognized as text. \pause
\item Written between single quotes, double quotes or triple quotes. \pause
\begin{verbatim}
>>> name='Dave'
>>> age='30'
>>> intro="Hi my name is "+name+".\nI'm "+age+" years old."
>>> intro
>>> print intro
>>> new_intro = """Hello!
... I'm Dave.
... What's up?"""
>>> new_intro
>>> print new_intro
\end{verbatim}
\end{itemize}
\end{frame}
%%%%%%%%%%%%%%%%%%%%%%%%%%%%%%%%%%%%%%%%%%%%%%%%%%

%%%%%%%%%%%%%%%%%%%%%%%%%%%%%%%%%%%%%%%%%%%%%%%%%%
%% Strings 2
%%%%%%%%%%%%%%%%%%%%%%%%%%%%%%%%%%%%%%%%%%%%%%%%%%
\begin{frame}[fragile]
 \frametitle{String}
 \begin{itemize}
 	\item You can call any character in the string. \pause
\begin{verbatim}
>>> intro[0]
>>> intro[1]
>>> intro[3]
\end{verbatim}
 \pause
 	\item Strings are immutable.  \pause
	\item But you can split a string into words. \pause
\begin{verbatim}
>>> intro.split()
\end{verbatim}
\pause
	 \item Or into any other chunks using a character. \pause
\begin{verbatim}
>>> new_intro.split('\n')
\end{verbatim} \pause
 \end{itemize}
\end{frame}
%%%%%%%%%%%%%%%%%%%%%%%%%%%%%%%%%%%%%%%%%%%%%%%%%%

%%%%%%%%%%%%%%%%%%%%%%%%%%%%%%%%%%%%%%%%%%%%%%%%%%
%% String 3
%%%%%%%%%%%%%%%%%%%%%%%%%%%%%%%%%%%%%%%%%%%%%%%%%%
\begin{frame}[fragile]
\frametitle{String}
\begin{itemize}
\item Run this code. What is happening?\pause
\end{itemize}
\begin{verbatim}
>>> intro[2:]
>>> intro[-2:]
>>> intro[:2]
>>> intro[:-2]
>>> intro[::2]
>>> intro[::-2]
>>> intro[::3]
\end{verbatim}
\end{frame}
%%%%%%%%%%%%%%%%%%%%%%%%%%%%%%%%%%%%%%%%%%%%%%%%%%

%%%%%%%%%%%%%%%%%%%%%%%%%%%%%%%%%%%%%%%%%%%%%%%%%%
%% String 4 
%%%%%%%%%%%%%%%%%%%%%%%%%%%%%%%%%%%%%%%%%%%%%%%%%%
\begin{frame}[fragile]
  \frametitle{String}
  \begin{itemize}
  	\item It requires a little more work to split a string into letters. \pause
	\begin{verbatim}
>>> [letter for letter in name]
>>> [letter for letter in intro]
\end{verbatim} \pause
	\item Let's combine them again. \pause
	\begin{verbatim}
>>> myletters=[letter for letter in intro]
>>> ''.join(myletters)
>>> '\n'.join(myletters)
\end{verbatim}
  \end{itemize}
\end{frame}
%%%%%%%%%%%%%%%%%%%%%%%%%%%%%%%%%%%%%%%%%%%%%%%%%%

%%%%%%%%%%%%%%%%%%%%%%%%%%%%%%%%%%%%%%%%%%%%%%%%%%
%% Int 
%%%%%%%%%%%%%%%%%%%%%%%%%%%%%%%%%%%%%%%%%%%%%%%%%%
\subsection{Int}
\begin{frame}[fragile]
  \frametitle{Int}
  \begin{itemize}
  	\item Integers. \pause
	\item You can do mathematical operations using these.
	\begin{itemize}
		\item Usual suspects: +  -  * /
		\item Exponentiate: **
		\item Remainder: \%		
	\end{itemize} \pause
	\item Remember the results are \emph{always}  rounded down! \pause
\footnotesize
\begin{verbatim}
>>> whole=5/3
>>> remainder=5%3
>>> "Five divided by three is %d and %d fifths" % (whole, remainder)
\end{verbatim}
\pause
	\item You can assign numbers using different operators. \pause
\footnotesize
\begin{verbatim}
>>> five=5
>>> five+=1
>>> five
>>> five/=3
>>> five
>>> five-=2
>>> five
\end{verbatim}
\end{itemize}
\end{frame}
%%%%%%%%%%%%%%%%%%%%%%%%%%%%%%%%%%%%%%%%%%%%%%%%%%

%%%%%%%%%%%%%%%%%%%%%%%%%%%%%%%%%%%%%%%%%%%%%%%%%%
%% Float
%%%%%%%%%%%%%%%%%%%%%%%%%%%%%%%%%%%%%%%%%%%%%%%%%%
\subsection{Float}
\begin{frame}[fragile]
  \frametitle{Float}
  \begin{itemize}
  	\item Real numbers. \pause 
	\item Written by adding the decimal to an integer. \pause
\begin{verbatim}
>>> 12.0/5
>>> float(7)
>>> type(2.*8)
\end{verbatim}	
  \end{itemize}
\end{frame}
%%%%%%%%%%%%%%%%%%%%%%%%%%%%%%%%%%%%%%%%%%%%%%%%%%

%%%%%%%%%%%%%%%%%%%%%%%%%%%%%%%%%%%%%%%%%%%%%%%%%%
%% List
%%%%%%%%%%%%%%%%%%%%%%%%%%%%%%%%%%%%%%%%%%%%%%%%%%
\subsection{List}
\begin{frame}[fragile]
\footnotesize
\frametitle{List}
\begin{itemize}
\item Collection of any type objects -- even lists \pause
\begin{verbatim}
>>> myletters
>>> type(myletters)
\end{verbatim} \pause
\item Lists can be changed, and include multiple object types \pause
\begin{verbatim}
>>> myletters.append(5)
>>> myletters[-1]
>>> type(myletters[-1])
>>> myletters[0]='Orange'
\end{verbatim}
\pause
\item Indexing starts at 0! \pause
\begin{verbatim}
>>> myletters[len(myletters)]
\end{verbatim}
\pause
\item You can insert into any position
\begin{verbatim}
>>> myletters.insert(2, '!')
\end{verbatim}
\pause
\item And remove from any position \pause
\begin{verbatim}
>>> myletters.pop(1)
\end{verbatim}
\end{itemize}
\end{frame}
%%%%%%%%%%%%%%%%%%%%%%%%%%%%%%%%%%%%%%%%%%%%%%%%%%

%%%%%%%%%%%%%%%%%%%%%%%%%%%%%%%%%%%%%%%%%%%%%%%%%%
%% Tuples
%%%%%%%%%%%%%%%%%%%%%%%%%%%%%%%%%%%%%%%%%%%%%%%%%%
\subsection{Tuples}
\begin{frame}[fragile]
\frametitle{Tuples}
\begin{itemize}
\item Tuples are like lists -- combination of any objects \pause
\item But are immutable \pause
\item Not very common, but very useful sometimes \pause
\begin{verbatim}
>>> tup=(1,6,5,'Apple')
>>> tup[1]
>>> tup[1]=9
>>> tup.append(9)
\end{verbatim}
\end{itemize}
\end{frame}
%%%%%%%%%%%%%%%%%%%%%%%%%%%%%%%%%%%%%%%%%%%%%%%%%%

%%%%%%%%%%%%%%%%%%%%%%%%%%%%%%%%%%%%%%%%%%%%%%%%%%
%% Dictionary
%%%%%%%%%%%%%%%%%%%%%%%%%%%%%%%%%%%%%%%%%%%%%%%%%%
\subsection{Dictionary}
\begin{frame}[fragile]
  \frametitle{Dictionary}
  \begin{itemize}
  	\item It is what it sounds like. \pause
	\item Here is how you create one. \pause
\begin{verbatim}
>>> myDict={'name':'Dave', 'last_name':'Carlson', 'age':30}
\end{verbatim}
	\item Unlike lists, there is no order to elements. \pause
	\item You call elements using keys. \pause
\begin{verbatim}
>>> myDict
>>> myDict.keys()
>>> myDict.values()
>>> myDict['last_name']
>>> myDict['middle_name']='George'
\end{verbatim}
\pause
\item These are particularly useful when we start defining classes (next class)
  \end{itemize}
\end{frame}
%%%%%%%%%%%%%%%%%%%%%%%%%%%%%%%%%%%%%%%%%%%%%%%%%%

%%%%%%%%%%%%%%%%%%%%%%%%%%%%%%%%%%%%%%%%%%%%%%%%%%
%% Conditionals
%%%%%%%%%%%%%%%%%%%%%%%%%%%%%%%%%%%%%%%%%%%%%%%%%%
\subsection{Conditionals}
\begin{frame}[fragile]
\frametitle{Conditionals}
\begin{itemize}
\item Perform an operation (or several) if condition is met (or not) \pause
\begin{verbatim}
>>> x=2
>>> if x==1:
...     print 'x is one'
... elif x==2:
...     print 'x is two'
... else:
...     print 'x is neither one nor two'
\end{verbatim}
\pause
\item Can be conditions or boolean (True or False) \pause
\item Multiple lines of code: \pause
\begin{itemize}
\item Indentation matters! \pause
\item Consistency is important, but exactly 4 spaces is `Pythonic' \pause
\item Will cause errors \pause
\item Even an empty line with spaces can cause errors
\end{itemize}
\end{itemize}
\end{frame}
%%%%%%%%%%%%%%%%%%%%%%%%%%%%%%%%%%%%%%%%%%%%%%%%%%

%%%%%%%%%%%%%%%%%%%%%%%%%%%%%%%%%%%%%%%%%%%%%%%%%%
%% Loops
%%%%%%%%%%%%%%%%%%%%%%%%%%%%%%%%%%%%%%%%%%%%%%%%%%
\section{Loops and functions}
\subsection{Loops}
\begin{frame}[fragile]
\frametitle{Loops}
\begin{itemize}
\item Two types of loops: for and while \pause
\item for loop: loops over some list \pause
\item while loop: loops while condition is true \pause
\item Can nest loops (and conditionals, etc.) \pause
\begin{verbatim}
>>> even_numbers=[]
>>> for i in range(1,10):
...     if i%2==0:
...         even_numbers.append(i)
...
>>> for letter in 'word': print letter
...
>>> sum([.05**i for i in range(1,10)])
>>> while len(myletters)>1:
...     myletters.pop()
...
\end{verbatim}
\end{itemize}
\end{frame}
%%%%%%%%%%%%%%%%%%%%%%%%%%%%%%%%%%%%%%%%%%%%%%%%%%

%%%%%%%%%%%%%%%%%%%%%%%%%%%%%%%%%%%%%%%%%%%%%%%%%%
%% Quick exercise
%%%%%%%%%%%%%%%%%%%%%%%%%%%%%%%%%%%%%%%%%%%%%%%%%%
\subsection{Exercise}
\begin{frame}
\frametitle{Quick exercise}
\begin{itemize}
\item Write code that saves the first ten numbers of the Fibonacci sequence to a list: \pause
\begin{itemize}
\item With a for loop \pause
\item With a while loop \pause
\end{itemize}
\item A while loop can always do what a for loop does, but syntax is simpler
\end{itemize}
\end{frame}
%%%%%%%%%%%%%%%%%%%%%%%%%%%%%%%%%%%%%%%%%%%%%%%%%%

%%%%%%%%%%%%%%%%%%%%%%%%%%%%%%%%%%%%%%%%%%%%%%%%%%
%% Functions
%%%%%%%%%%%%%%%%%%%%%%%%%%%%%%%%%%%%%%%%%%%%%%%%%%
\subsection{Functions}
\begin{frame}[fragile]
  \frametitle{Functions}
  \begin{itemize} 
  	\item They help write cleaner code. \pause
	\item Keep them simple. \pause
	\item You can return any type of object. \pause
	\item Don't forget to add \texttt{return} for output. \pause
\begin{verbatim}
>>> def addSquares(x,y):
...     return x**2+y**2
...
>>> addSquares(3,4)
\end{verbatim}
\pause
\item Change the Fibonacci code to find first $n$ numbers of sequence
  \end{itemize}
\end{frame}
%%%%%%%%%%%%%%%%%%%%%%%%%%%%%%%%%%%%%%%%%%%%%%%%%%


\end{document}